\documentclass{article}
\title{Comparison between constant and dynamic volatility scaling}
\author{ChatGPT}
\begin{document} 

\maketitle
\section{Data}
The study uses monthly prices from 55 global liquid futures instruments. The portfolio consists of 24 commodity contracts, 13 sovereign bond contracts, 9 currency contracts, and 9 equity index contracts. The data sources include Bloomberg, London Metal Exchange (LME), Intercontinental Exchange (ICE), Chicago Mercantile Exchange (CME), Chicago Board of Trade (CBOT), and K. French's website. The timeframe of the data is from June 1986 to May 2017. 
\section{Summary}
The paper compares two different approaches to scaling momentum strategies based on risk-adjusted measures. It analyzes the performance of constant and dynamic volatility scaling methods in a diversified portfolio of futures instruments. The findings show that the momentum strategy based on constant volatility scaling has a significantly higher alpha compared to dynamic volatility scaling, but this superiority becomes statistically insignificant during crisis and post-crisis periods. The study also identifies momentum crashes in futures markets, including during the 2007-2008 financial crisis. The paper emphasizes the importance of considering specific portfolio characteristics and datasets when implementing momentum strategies. It demonstrates the effectiveness of volatility scaling approaches in mitigating risk during crisis periods. The study further analyzes the performance of the momentum strategies during different sub-periods and provides regression analysis results. The constant volatility scaling approach has a higher alpha in the overall period, but the difference becomes statistically insignificant in crisis and post-crisis periods. The cumulative returns of the constant volatility scaling strategy are slightly higher than the dynamic volatility scaling strategy before 2003, but the difference expands between 2003 and 2007. The 2007-2008 financial crisis nearly eliminates the difference between the two scaled strategies, and after the crisis, the performance of both strategies improves with a small gap between them. Additionally, the paper compares the alphas of scaled XSMOM strategies with other scaled benchmark strategies and finds that the scaled XSMOM strategies significantly outperform the benchmark strategies. The volatility scaled strategies exhibit higher average returns and volatility than the corresponding unscaled strategies. The two scaled XSMOM strategies display the highest average returns, indicating greater profitability, but also higher uncertainty. The regression results suggest that constant volatility scaling produces statistically superior alphas compared to dynamic volatility scaling in most of the sample periods. The constant volatility scaling approach is identified as a more efficient volatility scaling method for momentum strategies in futures markets. One concern with the constant volatility scaling approach is that it displays higher risks compared to other volatility scaling approaches, which may affect its profitability in times of uncertainty. Future research could focus on investigating the source of this risk and how to alleviate it, such as exploring alternative ways to rank the winner/loser portfolio. The paper also references several relevant studies on momentum strategies and risk factors in financial markets, including Kilian (2009), Kim et al. (2016), Lewellen (2002), Miffre and Rallis (2007), Moskowitz and Grinblatt (1999), Moskowitz et al. (2012), Rouwenhorst (1999), Shen et al. (2007), Teplova and Mikova (2015), Wang and Xu (2015), and Zaremba and Szyszka (2016). 

\end{document}
\documentclass{article}
\title{Comparison between constant and dynamic volatility scaling}
\author{ChatGPT}
\begin{document} 

\maketitle
\section{Data}
The study uses monthly prices from 55 global liquid futures instruments. The portfolio consists of 24 commodity contracts, 13 sovereign bond contracts, 9 currency contracts, and 9 equity index contracts. The data sources include Bloomberg, London Metal Exchange (LME), Intercontinental Exchange (ICE), Chicago Mercantile Exchange (CME), Chicago Board of Trade (CBOT), and K. French's website. The timeframe of the data is from June 1986 to May 2017. 
\section{Summary}
The study compares the performance of two volatility scaling methods, constant volatility scaling (CVS) and dynamic volatility scaling (DVS), in momentum strategies. The authors perform momentum strategies based on these two approaches in a diversified portfolio consisting of 55 global liquid futures contracts. They also compare the results to time series momentum and buy-and-hold strategies.  \\
\\
The study finds that the momentum strategy based on the constant volatility scaling method (CVS) is the most efficient approach with an annual return of 15.3\%. The authors also compare the performance of the two approaches in different sub-periods and find that the superiority of CVS becomes statistically insignificant during crisis and post-crisis periods.  \\
\\
The study contributes to the literature by identifying the existence of momentum crashes in futures markets and demonstrating the reasonableness of employing volatility scaling approaches to avoid risks. It also finds that the CVS-based momentum strategy is more efficient and profitable than the DVS-based strategy. The study concludes that the CVS approach is a more efficient volatility scaling method for momentum strategies in futures markets. 

\end{document}